% !TEX root = ../thesis-example.tex
%
\pdfbookmark[0]{Abstract}{Abstract}
\chapter*{Abstract}
\label{sec:abstract}
\vspace*{-10mm}

Since the dawn of time, humans have sought to understand the nature and meaning of their dreams. However, despite millennia of philosophical speculation and more than a century of scientific exploration, several questions regarding dreams remain pending. How do we explain such intra- and inter-individuals differences in the frequency of dream recall? What rules govern the organization of dream content? What are the neural correlates of dreaming? What are the functions of dreaming? These opens questions are currently debated within the scientific community and are the subject of many hypothesis.

This lack of knowledge is partly due because dreaming occurs during sleep and is therefore not directly measurable, but solely accessible by recollection of the dreamer after awakening. In spite of this, dream research has been recently invigorated by the use of new technologies such as brain imaging.


\cleardoublepage

\pdfbookmark[0]{Résumé}{Résumé}
\chapter*{Résumé}
\label{sec:résumé}
\vspace*{-10mm}

Objet de nombreuses spéculations religieuses ou philosophiques depuis la nuit des temps, le rêve commence à être étudié de manière scientifique et expérimentale depuis le XIXème siècle. Phénomène fascinant et mystérieux, il reste encore l'une des grandes inconnues de la cognition humaine. Ce manque de connaissance à son propos s'explique en partie la vient en partie du fait que le rêve se déroule pendant le sommeil, et n'est donc pas mesurable directement.

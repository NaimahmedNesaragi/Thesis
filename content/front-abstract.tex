\addchap{Abstract}
\label{sec:abstract}
\vspace*{-10mm}

Humans have been intrigued by their dreams since the dawn of time. Yet, despite millennia of exploration, dreaming still remains one of the \emph{terra incognita} of the human cognition. The goal of the present thesis was to improve our understanding of this phenomenon through several studies, involving different methodologies and each addressing a particular aspect of dreaming. First, we investigated the mechanism of dream recall by comparing the cognitive, psychological and brain functioning of high and low dream recallers (HR and LR, respectively) during sleep and wakefulness (Studies 1 to 4). Second, we investigated the content and function of dreaming through an extensive behavioral analysis of the relationship between waking-life and dream content (Study 5). Finally, we leveraged our expertise in sleep science to develop an open-source and comprehensive software dedicated to sleep analysis (Study 6).

With regards to dream recall, our results revealed that the ability to recall dream is positively associated with a specific neurophysiological profile, characterized notably by a high activity in the default mode network during both sleep and wakefulness. For instance, we observed that, as compared to LR, HR exhibit a greater functional connectivity in regions critical to memory encoding just after awakening from sleep. HR also showed longer intra-sleep awakenings and higher scores of creative-thinking than LR. These findings led us to propose an integrative and comprehensive model of the dream recall process. Furthermore, the results of Study 5 enhanced and refined our comprehension of the continuity between waking-life and dream content, and provided support for the hypothesis of an active role of dreaming in emotional regulation. In conclusion, the experimental, theoretical and methodological contributions of the present work could serve as a basis for future research, in the hope that someday, we will be able to apprehend dreaming in all its richness and diversity.


\textbf{Keywords}: Dream, sleep, awakening, memory, magnetic resonance imaging, electroencephalography, brain networks, software development

\cleardoublepage

\addchap{Résumé}
\label{sec:résumé}
\vspace*{-10mm}

Objet de nombreuses spéculations religieuses, philosophiques et scientifiques, le rêve reste encore l'une des grandes \emph{terra incognita} de la cognition humaine. Le présent travail de thèse s’est attaché à améliorer notre compréhension de ce phénomène, en abordant ses multiples facettes à l’aide de méthodes comportementales et neuroscientifiques. Dans un premier temps, nous avons étudié les mécanismes du rappel du rêve en comparant le fonctionnement cognitif et cérébral de personnes se souvenant de leurs rêves très fréquemment (\emph{Rêveurs}) ou très rarement (\emph{Non-rêveurs}). Dans un deuxième temps, nous avons abordé la question de la fonction du rêve, en caractérisant notamment l’influence de la vie éveillée sur le contenu onirique. En parallèle de ces travaux, nous avons joué un rôle fondateur dans le développement d’un logiciel libre permettant la visualisation et l’analyse de tracés polysomnographiques de sommeil.

Nos résultats ont montré que la capacité à se souvenir de ses rêves est associée à un profil neurophysiologique et cognitif spécifique, caractérisé entre autre par une plus forte activité dans le réseau par défaut au cours de l’éveil, du sommeil, et dans les premières minutes suivant le réveil, période critique pour l’encodage du rêve en mémoire. Combinant ces résultats et la littérature existante, nous avons proposé un modèle intégratif du processus de mémorisation du rêve. Par ailleurs, nos observations comportementales ont permis d’améliorer notre compréhension de la continuité entre le contenu du rêve et la vie éveillée, et ont remarquablement suggéré que le rêve pourrait avoir un rôle actif dans des processus de régulation émotionnelle. En conclusion, de par ces apports expérimentaux, théoriques et méthodologiques, la présente thèse représente une avancée majeure dans la compréhension du rêve, et pourra servir, nous l’espérons, de support pour les futurs travaux visant à appréhender ce phénomène dans toute sa richesse et sa diversité.

\textbf{Mots-clés}: Rêve, sommeil, éveil, mémoire, imagerie par résonance magnétique, électroencéphalographie, réseaux cérébraux, développement logiciel

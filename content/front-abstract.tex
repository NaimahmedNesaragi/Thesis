\addchap{Abstract}
\label{sec:abstract}
\vspace*{-10mm}

Since the dawn of time, humans have sought to understand the nature and meaning of their dreams. However, despite millennia of philosophical speculation and more than a century of scientific exploration, several questions regarding dreams remain pending.

One question that constitutes the core problematic of this thesis relates to why there are such individual differences in the frequency of dream recall, or in other words, why some people remember up to several dreams per morning (High-recallers, HR) while some hardly ever recall one (Low-recallers, LR). To characterize the cerebral and behavioral correlates of this variability, we compared the sleep microstructure (Study 1), as well as the brain functional connectivity in the minutes following awakening from sleep, a period marked by sleep inertia (Study 2). Among other results, we have shown that just after awakening, HR demonstrated a greater functional connectivity within regions involved in memory processes (default mode network). We proposed that this reflect a differential neurophysiological profile, which could facilitate in HR the retrieval of dream content upon awakening. Second, the numerous answers to the recruitment questionnaire of this study allowed us to conduct an epidemiological survey to characterize the sleep and dream habits of a large sample of French college students from Lyon 1 University (Study 3).

In another study, we focused on the relationships between waking-life and dream content (Study 4). Our results enhanced and refined our comprehension of the factors influencing the likelihood of incorporation of waking-life elements into dreams, and provided support for the hypothesis of an active role of dreaming in emotional regulation.

Lastly, we designed a free and open-source software dedicated to the visualization and analysis of polysomnographic recordings (Study 5), which aims at providing an intuitive and portable graphical interface to students and researchers working on sleep.

\textbf{Keywords}: Dream, sleep, awakening, memory, magnetic resonance imaging, electroencephalography, brain networks, software development

\cleardoublepage

\addchap{Résumé}
\label{sec:résumé}
\vspace*{-10mm}

Objet de nombreuses spéculations religieuses ou philosophiques, le rêve reste encore l'une des grandes terra incognita de la cognition humaine.

Une des questions récurrentes sur le rêve porte sur la grande variabilité de fréquence de rappel de rêve. En effet, alors que certaines personnes se souviennent de leurs rêves quotidiennement (« Rêveurs »), d’autres ne s’en souviennent que très rarement (« Non-rêveurs »). Le principal objectif de notre travail de thèse a été de caractériser les corrélats cérébraux et comportementaux de cette variabilité interindividuelle, en comparant entre ces deux groupes la structure du sommeil (Étude 1), mais aussi l’activité cérébrale pendant les minutes qui suivent le réveil (Étude 2). Nous avons entre autres montré que les « Rêveurs » faisaient preuve d’une plus grande connectivité fonctionnelle au sein du réseau par défaut et de régions impliquées dans des processus mnésiques dans les minutes suivant l’éveil, ce qui pourrait faciliter chez ces personnes le rappel et/ou la consolidation du rêve. Cette étude nous a également permis, grâce à l’analyse des nombreuses réponses obtenues au questionnaire de recrutement, de mesurer les habitudes de sommeil et de rêve chez un échantillon large d’étudiants de l’Université de Lyon 1 (Étude 3).

Dans une quatrième étude comportementale, nous nous sommes intéressés au lien existant entre la vie éveillée et le contenu du rêve. Nos résultats ont permis de mieux caractériser les facteurs influençant la probabilité d’incorporation des évènements de la vie éveillée dans le rêve, et ont mis en évidence l’importance du rêve dans des processus de régulation émotionnelle.

Finalement, en parallèle de ces travaux, nous nous sommes attachés au développement d’un logiciel gratuit de visualisation et d’analyse de tracés de polysomnographie, dont l’objectif est de fournir une interface intuitive et portable aux étudiants et chercheurs travaillant sur le sommeil.

\textbf{Mots-clés}: Rêve, sommeil, éveil, mémoire, imagerie par résonance magnétique, électroencéphalographie, réseaux cérébraux, développement logiciel

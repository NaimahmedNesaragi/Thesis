\addchap{Acknowledgement}
\label{sec:acknowledgement}
\vspace*{-10mm}

Je tiens avant toute chose à remercier ma directrice de thèse, Perrine Ruby, dont la confiance et le soutien ont rendu possible ce travail. Elle sait, je l'espère, toute la considération et l'amitié que j'ai pour elle, et je ne la remercierai jamais assez de m'avoir toujours poussé vers l'avant, en me donnant les moyens par exemple de participer à de nombreux colloques internationaux. Sa rigueur et son honnêteté scientifique, ainsi que son engagement indéfectible pour ses étudiants, sont autant de qualités que je souhaiterai avoir s'il m'est donné un jour d'encadrer à mon tour des étudiants.

Ma pensée se tourne ensuite vers tous les membres du laboratoire DYCOG avec qui j’ai vécu et partagé des moments inoubliables tout au long de ces quatre dernières années. Sans pouvoir citer, par soucis écologique, toutes les personnes avec qui j’ai échangé, je tiens tout de même à évoquer mes collègues et amis du bureau d’en haut, Laurie-Anne, Kony, Etienne, Enrico, Stefano, dont la présence au Vinatier ne relève sans doute pas d’un hasard total ; les copains du café de 7 h et du saucisson de 19 h, Florian, Benoit, Thibault ; l’équipe des méditants Croix-roussien, Kristien, Antoine, Jelle, Oussama, pour ces moments de rire au milieu des fameux bouchons Lyonnais ; et bien sûr tous les autres doctorants, étudiants, ingénieurs, chercheurs, et personnels administratifs, qui de par leur bonne humeur et leur expertise ont rendu cette thèse si agréable.

Toute ma gratitude se porte également aux participants de nos expériences, d’imagerie ou de comportement, qui sont de fait la composante essentielle de ce travail. Ils nous ont prêté leurs rêves et leurs cerveaux, tout en gardant une motivation sans faille et un sourire constant.

Je pense ensuite à mes amis de longue date, Ylane, Caroline, Olivier, Géraud, Alex, Matthieu, Bertrand, qui ont su égayer ma vie durant toutes ces années, à coup de rires, de délires et de musique.

Le meilleur pour la fin dit-on - je remercie du fond du cœur mes parents, Isabelle et Alain ainsi que ma sœur Amélie et sa petite famille grandissante, sans qui tout cela n'aurait jamais été possible. Ma pensée finale se tournera vers Alisé, qui en plus d’être un sujet idéal pour étudier le sommeil au quotidien, est la personne avec qui je souhaite construire mes rêves.

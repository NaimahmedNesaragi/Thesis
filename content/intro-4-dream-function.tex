\cleardoublepage

\chapter{Dream function}
\label{sec:dream-function}

\cleanchapterquote{J’ai rêvé tant et plus, mais je n’y entends note.\endnote{English translation: \q{Dreamed indeed I have, and that right lustily; but I could take along with me no more thereof that I did goodly understand.} ---François Rabelais. Pantagruel. 1532}}{François Rabelais}{Pantagruel, 1532}

\section{Historical perspective}
\label{sec:dream-func:history}

\subsection{Ancient and classical history}
\label{sec:dream-func:history:ancient}

Since the dawn of times, humans have tried to assign meaning to their dreams. In many ancient civilizations, dreams were considered as omens or messages from deities, and needed therefore to be correctly interpreted. Numerous examples of dreams sent by Gods can be found in Mesopotamian, Egyptian and Greek mythological narratives, as well as in the sacred books of the three main monotheistic religions (see \citealp{de_koninck_sleep_2012}). Greek philosophers, however were the first to consider dreaming as a natural phenomenon. They provided different explanations of the nature and meaning of dreams, some of which are well in tune with modern dream research. For example, anticipating the notion of continuity between waking and dreaming, Cicero and Herodotus believed that dreams are produced by thoughts, concerns and conversations a dreamer had during the preceding days. Plato on the contrary viewed dreams as the expression of hidden desires and intolerable behaviors, an idea consistent with Freud’s repression hypothesis. Finally, Aristotle thought that dreams were caused by external and internal bodily sensations, an idea consistent with Hobson’s stochastic theory of dream generation.

\subsection{The royal road to the unconscious}
\label{sec:dream-func:history:psychanalysis}

The father of psychanalysis viewed dreams as the \q{royal road to the unconscious} \citep{freud_interpretation_1900}. He defined the unconscious as a part of our mind made up of thoughts, desires, emotions, and knowledge that we are unaware of, but that nevertheless profoundly influence and guide our behaviors. Freud believed that the ego’s defenses are lowered during dreaming, which allows the unconscious mind and the repressed material (i.e. material made unavailable to consciousness because morally unacceptable) to come through awareness, albeit in a distorted form to avoid sleep disturbances, hence his famous aphorism that \q{dreams are the guardians of sleep}. For him, the dream is formed of the manifest content (i.e. the dream as the dreamer recalls it), which is often based on mundane and insignificant day-residues, and the latent content (i.e. symbolic meaning of the dream). The latent content of the dream can be extract from the manifest content using free-association in order to unravel the unconscious thoughts expressed in the dream. Therefore, in Freud’s model, the dream need to be explicitly remembered and interpreted to possess an adaptive value (thought the sleep protection mechanism might be adaptive \emph{per se}). It is noteworthy that his hypothesis \q{has rarely been considered by neuroscientists who often consider Freud’s work and theory unscientific} \citep{ruby_experimental_2011}. Yet, this issue of whether Freud's theory is scientifically valid, i.e. with experimentally testable predictions and the possibility to falsify it \citep{popper_conjectures_2014}, has been recently addressed by \citet{guenole_dreams_2013} who concluded that \q{Freud's theory of the basic function of dreaming is empirically testable ... and can be considered as a valuable contribution to the scientific knowledge}.

\subsection{Psychological individualism}
\label{sec:dream-func:history:jouvet}

Michel Jouvet, one of the pioneers of sleep research, co-discoverer of REM sleep, was also greatly interested in dreams. He kept a dream diary for years, which he used to describe several quantitative measures of dream content. For instance, he was one of the first to report the dream-lag effect \citep{jouvet_memoire_1979}. With regards to the function of dreaming, which he equated at the time to the function of REM sleep, he first proposed at the beginning of his career that it allowed the maintenance of typical behaviors of species, an idea that stems from his own findings on the complex motor sequences exhibited by cats during REM sleep after suppressing muscular atonia (see section \ref{sec:dream-research:link:rem-sleep}). Later on, he modified this theory and proposed that dreaming is in fact a kind of iterative neurological programming system whose aim is to preserve the expression of the genetic program that codes for psychological characteristics. According to him, this process would ensure the stability of personality traits across time \citep{jouvet_sommeil_1991}.

\section{Modern theories}
\label{sec:dream-func:modern}

\subsection{An epiphenomenom of REM sleep}
\label{sec:dream-func:modern:nofunc}

Based on the neurophysiological properties of REM sleep, which he equated with dreaming, Alan Hobson proposed that dreaming is an epiphenomenon of REM sleep \citep{hobson_dream_1998}. According to him, the dream imagery is the result of cortical centers trying to create meaning from brainstem-driven signals generated during REM sleep (the so-called activation-synthesis model). In this theory, the dream content is therefore stochastic and is very unlikely to represent an adaptive advantage. It should be noted, however, as noticed by \citet{windt_dreaming:_2015}, that \q{Hobson does not deny that dreams can have meaning and can reflect the personality and concerns of the dreamer. He just thinks that their meaning is transparent and immediately obvious to the dreamer, rather than requiring an elaborate process of interpretation}.

\subsection{Threat / Social simulation theory}
\label{sec:dream-func:modern:revonsuo}

\citet{revonsuo_reinterpretation_2000} proposed that dreaming is a virtual reality in which the dreamer can simulate threatening events and therefore be better prepared to face upcoming dangers in waking life (the so-called threat simulation theory, TST). According to him, \q{dream consciousness is essentially an ancient biological defense mechanism, evolutionarily selected for its capacity to repeatedly simulate threatening events} \citep{valli_threat_2005}. As such, dream content is more consistent with the original evolutionary environment of the human species (e.g. high level of violence and intergroup aggression between males) rather than the present one, and this could for example explain that the most frequent type of social interaction found in dreams, especially in males, is aggression \citep{hall_content_1966}. \citet{valli_threat_2005} further tested this hypothesis by analyzing the content of dream reports from severely traumatized and non-traumatized children. As predicted by the theory, the reported dreams of severely traumatized children included a higher number of threatening events than those of non-traumatized children. These threats were also more severe in nature than the threats of non-traumatized children.

The same team has recently proposed that, more than a simulation of threats, dreaming is a global simulation platform, with a strong focus on social perception and interactions (the so-called social simulation theory, SST; \citealp{revonsuo_avatars_2015}), which are, from an evolutionary standpoint, as relevant as threats (it is now well accepted that the social environment has afflicted strong selection pressures on human cognition). As this theory is very new, its main predictions remain to be experimentally confirmed or refuted.

\subsection{Memory consolidation}
\label{sec:dream-func:modern:memory}

There is converging evidences from both animal and human research that sleep optimizes and consolidates the memory of newly acquired information \citep{rasch_odor_2007, diekelmann_memory_2010}. Based on this, a current hypothesis in dream research is that dreaming in itself is related to sleep-dependent memory consolidation (review in \citealp{schredl_is_2017}).
This proposal was tested in \citet{wamsley_dreaming_2010} in a study where 50 subjects were trained on a virtual navigation task before taking a 45 min nap. Remarkably, subjects who dreamed about the task had better post-nap performances than subjects who did not dream. However, as only 4 out of 50 subjects actually dreamed about the task (among which two reports just included hearing the music presented during the training session), the statistical power of this study is very low and it would seem premature to draw conclusions from this single finding. The same year, \citet{schredl_is_2010} investigated whether dream characteristics are related to the over-night improvement of a mirror tracing task (i.e. the participants must trace different figures they only saw in a mirror). They were unable to find an effect of direct incorporations of the mirror tracing task into the dream on over-night improvement. It should be noted however that, again, the rate of direct incorporation of the task was very low, which consequently results in a low statistical power. Another methodological bias is that these studies focused only, and for obvious reason, on recalled dreams and thus omit a large fraction of non-recalled dreams. The role of the different sleep stages in the putative role of dreaming in memory consolidation needs to be further investigated. To sum up, these results are hitherto inconclusive on the role of dreaming in memory consolidation.

\subsection{Emotional regulation}
\label{sec:dream-func:modern:emotion}

Cartwright proposed that dreaming is involved in emotional regulation \citep{cartwright_role_1998, cartwright_role_1998-1}. She reached this conclusion after observing that, in healthy subjects, the depression level before sleep was significantly correlated with affect in the first REM report. In the same study, she also observed that low scorers on the depression scale displayed a flat distribution of positive and negative affect in dreams, whereas those with a depressed mood before sleep showed a pattern of decreasing negative and increasing positive affect in dreams reported from successive REM periods. Secondly, she observed that among individuals who were depressed following a divorce, those who reported more negative dreams early in the night and fewer at late-night were more likely to be in remission one year later, compared to subjects in which this pattern was inverted. From these two works, she proposed that dreaming may actively moderate mood overnight in healthy individuals, with the decreasing rate of negative dreams across the night reflecting a within-sleep emotional regulation process.

Linking emotional regulation and memory consolidation processes, \citet{perogamvros_roles_2012} recently proposed the Reward Activation Model according to which emotionally relevant experiences (including threat-related information) have a higher probability of being activated during sleep and have a preferential access to sleep-related memory consolidation processes. According to them, one of the main functions of dreaming is \q{to expose the sleeper to rewarding or aversive stimuli, in order to maintain and improve offline memory consolidation processes and performance in real life situations, while also contributing to emotion regulation processes} \citep{meerlo_sleep_2013}.

\subsection{Summary}
\label{sec:dream-func:modern:summary}

Despite several decades of scientific research on dream content, there is still no consensus on whether dreaming serves a function or not. As \citet{blagrove_distinguishing_2011} stated in the summary of his chapter on dream function, \q{There is to my knowledge no evidence that dreaming has a functional effect, or is associated with any brain process that is having a functional effect, as the literature on the supposed consequences of particular dream imagery is composed of correlational studies}. Further research is therefore needed to move forward on this issue, notably by experimentally testing the main predictions of these theories in order to confirm or refute them.
